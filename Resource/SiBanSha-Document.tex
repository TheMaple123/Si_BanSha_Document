\documentclass[UTF-8]{ctexart}

\title{四班杀文档(草案)}
\author{警告!该版本尚未完成!}
\date{2021.3.14 \space 编辑}

\begin{document}

\maketitle
\thispagestyle{empty}
\newpage

\begin{center}
\section*{一、前言}
\end{center}

\subsection*{1.1    介绍}
这是一个基于三国杀改编,与四班的梗相关,并时不时增加网络新梗的奇怪东西

{\noindent} \rule[-10pt]{33em}{0.05em}
\begin{center}
\subsection*{1.2    工作人员}

    \subsubsection*{策划组}
    林\rule[-2pt]{1em}{1em}旭 \hspace{1.0cm} 吴\rule[-2pt]{1em}{1em}陈
    \subsubsection*{美术组}
    吴\rule[-2pt]{1em}{1em}陈
    \subsubsection*{手工组}
    华\rule[-2pt]{1em}{1em}平(组长) \\ 姜\rule[-2pt]{1em}{1em}俊(副组长) \hspace{1.0cm} 李\rule[-2pt]{1em}{1em}衡(组员)

\end{center}
{\noindent} \rule[-10pt]{33em}{0.05em}

\newpage
% 实际上,你可以在这个 \newpage 之前在写点东西(第二页没写满);后面就是规则和设定集了(LaTeX真好用)

\section*{三、设定}

\subsection*{(一)   手牌}

\begin{flushleft}
1.杀 \footnote{若无特殊说明,则为默认,下文同}  \hspace{1.0cm} 2.闪 \hspace{1.0cm} 3.桃 \hspace{1.0cm} 4.酒---哇水
\end{flushleft}

\subsection*{(二)   锦囊牌}

\begin{flushleft}
1.无懈可击---有懈可击 \hspace{0.2cm} 2.万箭齐发---作业齐发 \hspace{0.2cm} 3.南蛮入侵---文英入侵\\
4.顺手牵羊---乐迪行为 \hspace{0.22cm} 5.决斗---乃强对决 \hspace{1cm} 6.过河拆桥---阿炜行为\\
7.无中生有 \hspace{2.05cm} 8.桃园结义---食堂结义 \hspace{0.2cm} 9.五谷丰登---有福同享\\
10.铁索连环---有难同当 \hspace{0.1cm} 11.乐不思蜀
\end{flushleft}

\subsection*{(三)   装备牌}

\begin{flushleft}
1.诸葛连弩---粉笔 \hspace{0.8cm} 2.青釭剑---钢尺 \hspace{1.5cm} 3.贯石斧---三角板\\
4.雌雄双股剑---圆规 \hspace{0.4cm} 5.青龙偃月刀---拖把 \hspace{0.8cm} 6.丈八蛇矛---竹鞭\\
7.麒麟弓---附魔弓 \hspace{0.8cm} 8.八卦阵---班优 \hspace{1.5cm} 9.寒冰剑---雨伞\\
10.仁王盾---畚斗 \hspace{1cm} 11.方天画戟---扫把\\
13.+1 马:爪黄飞电---神舟五号 \hspace{0.1cm} 绝影---电动车 \hspace{0.1cm} 的卢---集结号\\
紫骍---自然选择
\end{flushleft}

\subsection*{(四)   身份牌}

\begin{flushleft}
1.主公---班主任 \hspace{0.4cm} 2.忠臣---班长 \hspace{0.4cm} 3.内奸---好生 \hspace{0.4cm} 4.反贼---差生\\
\end{flushleft}

\newpage

\subsection*{(五)   事件}

\begin{flushleft}
1.上课 \space 本回合不能摸牌
\end{flushleft}

\newpage
% 下面就是武将了

\subsection*{(六)   武将牌}
梅子(师) \space HP \footnote[1]{指血量,下文同} :3\\
{\noindent}1.物理大师 \space 使用物理公式废弃他人一张手牌\\
{\noindent}2.春秋笔法 \space 阴阳怪气学生并扣它一滴血,摸一张牌 \footnote[2]{无特殊说明则作用对象为自己,下文同} \\
{\noindent}3.科技大师 \space 使用尖端科技锁定对方两回合\\
{\noindent}4.``我'' \space 将对方两张牌 \footnote[3]{除武将、血量、身份牌,注意与手牌等的区别} 据为己有,锁定他一回合并扣一滴血


\end{document}